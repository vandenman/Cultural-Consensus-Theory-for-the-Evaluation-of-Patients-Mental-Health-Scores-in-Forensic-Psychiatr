\documentclass{article}
\usepackage[utf8]{inputenc}

% packages
\usepackage{amsmath,amsfonts,amssymb, bm}
\usepackage{graphicx}
\usepackage[colorlinks=true, allcolors=blue]{hyperref}
\usepackage{apacite}
\usepackage{authblk}  % for authors
\usepackage{caption}
%\usepackage{setspace} % doublespacing
\usepackage{subcaption}
\usepackage{booktabs}
\usepackage{nicefrac}

\usepackage{color}
\usepackage{todonotes}

\newcommand{\EJ}[1]{\todo[inline, color=green]{EJ: {#1}}}
\newcommand{\DON}[1]{\todo[inline, color=white]{Don: {#1}}}


\title{Cultural Consensus Theory for the Evaluation of Patients’ Behavior in Psychiatric  Detention Centers}

\renewcommand{\thefootnote}{\fnsymbol{footnote}}
\author[1]{Don van den Bergh\thanks{Correspondence concerning this article should be addressed to:  
\\  Don van den Bergh 
\\  University of Amsterdam, Department of Psychological Methods
\\  Postbus 15906, 1001 NK Amsterdam, The Netherlands
\\  E-Mail should be sent to: donvdbergh@hotmail.com.}}
\author[1]{wie nog meer?}
\author[1]{Eric-Jan Wagenmakers}
\affil[1]{University of Amsterdam}
\date{}

\begin{document}

\maketitle

\begin{abstract}
In many psychiatric detention centers, patients' mental health is monitored at regular intervals. Typically, clinicians score patients using a Likert scale on multiple criteria including hostility. Having an overview of patients’ scores benefits staff members in at least three ways. First, the scores may help adjust treatment to the individual patient; second, the change in scores over time allow an assessment of treatment effectiveness; third, the scores may warn staff that particular patients are at high risk of turning violent. Practical importance notwithstanding, current practices for the analysis of mental health scores are suboptimal: evaluations from different clinicians are averaged (as if the Likert scale were linear and the clinicians identical), and patients are analyzed in isolation (as if they were independent). Uncertainty estimates of the resulting score are often ignored. Here we outline a quantitative program for the analysis of mental health scores using cultural consensus theory (CCT; \citeNP{Anders2015cultural}). CCT models take into account the ordinal nature of the Likert scale, the individual differences among clinicians, and the possible commonalities between patients. In a simulation, we compare the predictive performance of the CCT model to the current practice of aggregating raw observations and, as a more reasonable alternative, against often-used machine learning toolboxes. In addition, we outline the substantive conclusions afforded by application of the CCT model. We end with recommendations for clinical practitioners who wish to apply CCT in their own work. 
\end{abstract}

\newpage

% Introduction

Psychiatric detentions centers monitor the mental health of their patients at regular intervals. A clinician, psychiatrist or other staff member scores a patient on multiple criteria. Next, these ratings of patients' mental health may be used for a variety of purposes. For instance, the scores may help adjust treatment to the individual patient; second, the change in scores over time allow an assessment of treatment effectiveness; third, the scores may warn staff that particular patients are at high risk of turning violent. 

Current practice of averaging ratings across raters is suboptimal.

CCT is a model capable of separating item, rater, and patient characteristics.

Here, we introduce a quantitative procedure for the analysis of rated mental health scores. 

\section*{Cultural Consensus Theory}

The Cultural Consensus Model as defined in \cite{Anders2015cultural}




personen/ raters zijn gebiased. Kan niet alleen eigenschappen van de gedetineerden halen maar ook die van de raters.

Covariaat voor e.g., hulpverleners versus psychiaters. Meerdere groepen raters (fixed effect).

Hierachisch niveau over patienten.

Covariaat voor groepen gedetineerden (fixed effect misdrijf).
Ofwel order restrictie voor deze groepen.

Missing values?

Hoe combineren we verschillende schalen van items?

kaart van hoe ontwikkelt zich dit over de tijd (zie figuur in proposal)

TODO: change model in NWO proposal to mimic SDT approach.

\section*{Simulation Results}


\section*{Discussion}

Aanbeveling voor de praktijk

Bijhouden van evaluaties (scores + rater), liefst met hoge frequentie. 
Reden waarom iemand opgesloten is (reason of incarceration).
zo min mogelijk missing not at random.
``handig'': training in invullen om verschillen tussen raters te minimalizeren.
(V)AR component?

\end{document}
